\chapter{Grundläggande sannolikhetsteori}

\section{Kombinatorik}
För att kunna räkna ut sannolikheten för en händelse behöver vi kunna räkna ut antalet möjliga utfall. Detta görs med hjälp av kombinatorik.\\

\subsection{Multiplikationsprincipen}
Antag ett slumpexperiment i $k$ steg. Låt $n_j$ vara antalet möjliga utfall i $j\in\{1,\dots,k\}$-te steget. Då ges antalet utfall för hela experimentet av:
$$
	\Pi^k_{j=1}n_j=n_1\cdot n_2\cdot\ldots\cdot n_k
$$
\begin{exempel}{Exempel}
	På en dans var 8 herrar och 9 damer bjudna. Om herrarna bara dansar med damerna och vise versa,\\ hur många olika danspar kan det bli?
\end{exempel}

\subsection{Permutation}
Hur många tal kan vi bilda med siffrorna ${1,2,3}$ utan att repetera en siffra?\\
Enligt multiplikationsprincipen finns $3\cdot2\cdot1=6$ möjliga kombinationer:
$$
	(1,2,3),(1,3,2),(2,1,3),(2,3,1),(3,1,2),(3,2,1)
$$

För att få fram antalet permutationer av $n\in\mathbb{N}$ element kan vi använda följande:\\
När vi väljer första objektet har vi $n$st val, för andra har vi $(n-1)$st val och så vidare. Antalet ordningar är då:
$$
	n(n-1)(n-2)\cdot\ldots\cdot3\cdot2\cdot1=\prod^n_{i=1}=n!
$$
$n!$ kallas $n-$fakultet och är ett sätt att räkna antalet permutationer av $n$ element.\\
$0-$fakultet definieras som $0!=1$.

\subsection{Laplace experiment}
Antag att vi har ett slumpmässigt försök med ändligt många möjliga utfall, där varje utfall är lika troligt, då säger man att sannolikheten är likformigt fördelad. Sannolikheten för en händelse $A$ är då:
$$
	P(A)=\frac{\|A\|}{\|S\|}
$$
Där $\|M\|$ betecknar antalet element i mängden $M$.

\section{Räkneregler}
\begin{itemize}
	\item \textbf{Komplement}: Komplementhändelsen till $A$ definieras som $A'=\text{''A inträffar inte''}$
	\item \textbf{Additivitet}: Om $A$ och $B$ är disjunkta gäller $\mathbb{P}(A\cup B) =\mathbb{P}(A)+\mathbb{P}(B)$
	\item \textbf{Additionssatsen}: För godtyckliga händelser $A,B$ gäller $\mathbb{P}(A\cup B)=\mathbb{P}(A)+\mathbb{P}(B)-\mathbb{P}(A\cap B)$
\end{itemize}

\section{Betingad sannolikhet}
\begin{exempel}{Maskiner}
	I en tillverkningsprocess för en viss produkt kontrolleras kvaliteten, och en produkt klassificeras för enkelhets skull som antingen duglig eller defekt. Efter ett maskinbyte har vi tillgång till data om antalet defekta såväl innan som efter bytet.
	$$
		\begin{matrix}
			~                   & \text{Duglig} & \text{Defekt} & \vline & \text{Totalt} \\
			\text{Äldre maskin} & 170           & 10            & \vline & 180           \\
			\text{Ny maskin}    & 115           & 5             & \vline & 120           \\
			\hline
			\text{Totalt}       & 285           & 15            & \vline & 300
		\end{matrix}
	$$
	$A-$''Slumpvis vald produkt är duglig''.\\
	$B-$''Slumpvis vald produkt är tillverkad med äldre maskin''.\\
	$C-$''Slumpvis vald produkt är duglig, givet tillverkad med äldre maskin''.\\
	$\mathbb{P}(A)=\frac{285}{300}$,\\ $\mathbb{P}(B)\frac{180}{300}$,\\ $\mathbb{P}(C)=\mathbb{P}(A\vline B)=\frac{170}{180}=\frac{\mathbb{P}(A\cap B)}{\mathbb{P}(B)}$
\end{exempel}
För två händelser $A$ och $B$ där $\mathbb{P}(B)>0$ definieras den betingade snnolikheten för $A$, givet att händelsen $B$ inträffar genom:$$
	\mathbb{P}(A\|B)=\frac{\mathbb{P}(A\cap B)}{\mathbb{P}(B)}
$$
\section{Multiplikationsregel och Bayes sats}
\subsection{Multiplikationsregel}
För två händelser $A$,$B$ gäller$$
	\mathbb{P}(A\cap B)=\mathbb{P}(A\|B)\mathbb{P}(B)=\mathbb{P}(B\|A)\mathbb{P}(A)
$$
\subsection{Bayes sats}
$$
	\mathbb{P}(A\|B)=\frac{\mathbb{P}(B\|A)\mathbb{P}(A)}{\mathbb{P}(B)}
$$
