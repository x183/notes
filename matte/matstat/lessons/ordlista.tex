\chapter{Ordlista}
\subsubsection{Utfall}
Resultatet av ett slumpmässigt försök eller experiment.

\subsubsection{Utfallsrummet}
Mängden av alla möjliga utfall.

\subsubsection{Trädiagram}
Sätt att beskriva ett utfallsrum av stegvisa slumpexperiment.

\subsubsection{Händelse}
En delmängd av utfallsrummet $S$.

\subsubsection{Omöjliga händelsen}
Annat namn för den tomma mängden $\emptyset$.

\subsubsection{Säkra händelsen}
Mängden $S$ kallas för den säkra händelsen.

\subsubsection{Disjunktion/ oförenlighet}
Två händelser $A$ och $B$ är disjunkta om de inte har några gemensamma utfall. Detta skrivs $A \cap B = \emptyset$.

\subsubsection{Parvis disjunktion/oförenlighet}
Två händelser $A$ och $B$ är parvis disjunkta om de inte har några gemensamma utfall. Detta skrivs $A \cap B = \emptyset$ $\forall i\neq j$.

\subsubsection{Kombinatorik}
Teorin om räkning av möjliga utfall (kombinationer).

\subsubsection{Permutation}
En ordning av element i en mängd kallas för en permutation av elementen i mängden.
\subsubsection{Stokastisk variabel}
En stokastisk variabel är en funktion som för varje utfall i ett slumpmässigt försök antar ett reellt tal, $X:\Omega\rightarrow\mathbb{R}$
\subsubsection{Diskret stokastisk variabel}
En stokastisk variabel som antar ett ändligt/uppräkneligt antal värden.

\subsubsection{Kontinuerlig stokastisk variabel}
En stokastisk variabel som antar alla värden i ett intervall.