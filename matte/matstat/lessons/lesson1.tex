\chapter{Diskreta stokastiska variabler}

\section{Föregående föreläsningen}
Bayes sats
$P(A\|B)=\frac{P(A\cap B)}{P(B)}=\frac{P(B\|A)P(A)}{P(B)}$
\\
$\leftarrow P(A\|B)=P(A)\Leftrightarrow P(A\cap B)=P(A)P(B)$\\
$\frac{P(A\cap B)}{P(B)}$

\section{Stokastiska variabler}
Binomialfördelade och Geometriskt fördelade är de som är viktigast idag.\\
\subsection{Vad är en stokastisk variabel?}
En stokastisk variabel $X$ är en funktion som för varje utfall i ett slumpmässigt försök antar ett reellt tal, dvs. $X:\Omega\rightarrow\mathbb{R}$
\\
\subsubsection{Diskreta stokastiska variabler}
En stokastisk variabel som antar ett ändligt antal/uppräkneligt antal värden
\subsubsection{Kontinuerlig stokastisk variabel}
Kan istället anta värden i ett intervall.
\subsection{sannolikhetsfunktion}
Kan vara:
\begin{itemize}
\item Antalet ögon vid tärningskast
\item Antalet gånger vi måste kasta ett mynt innan vi får krona
\end{itemize}

$$f(x)=P(X =x),x\in\mathbb{R}$$
En funktiin är en sannolikhetsfunktion om och endast om\\
$f(x)\geq0$
\\
$\sum_{x\in X(\Omega)}f(x)=1$\\
$F(x)=P(X\leq x)$ kallas fördelningsfunktionen till X.