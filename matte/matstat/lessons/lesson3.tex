\chapter{Fler användbara fördelningar}
\section{Negativ binomialfördelning}
Negativ binomialfördelning beskriver hur många gånger vi behöver upprepa ett försök tills $r$ försök har lyckats.
Antag att försöken är identiska ,oberoende av varandra, och lyckas med sannolikhet $p$, då har $X$ en negativ binomialfördelning med parametrarna $r$ och $p$, och motsvarande sannolikhetsfunktion ges av $$
	f(k)=P(X=k))\left(^{k-1}_{r-1}\right)p^r(1-p)^{k-r}\text{ , }k\geq r
$$
$X\sim NB(r,p)$ Då kan vi skriva $X=X_1+X_2+\ldots+X_r$ där $X_1,\ldots,X_r\sim\text{geom}(p)$
\begin{exempel}{Exempel}
	Vad är sannolikheten att vi får en sexa först på det tredje kastet?\\
	\textbf{Lösning:}\\
	$$\text{geom}(\frac{1}{6})$$
\end{exempel}

\section{Hypergeometrisk fördelning}
Antag att vi har $N$ objekt, varav $r$ har en egenskap vi tycker om. Vi väljer $n$ objekt slumpmässigt och låter $X$ vara antalet av de valda som har egenskapen. Då har $X$ en hypergeometrisk fördelning med parametrarna $N$, $r$ och $n$. Sannolikhetsfunktionen ges av $$
	f(k)=P(X=k)=\frac{\left(^{r}_{k}\right)\left(^{N-r}_{n-k}\right)}{\left(^{N}_{n}\right)}
$$
\subsection{Väntevärdet och variansen}
\subsubsection{Väntevärde}
$$\mathbb{E}[X]=\frac{nr}{N}$$

\subsubsection{Varians}
$$
	\text{Var}(X)=\frac{nr(N-n)(N-r)}{N^2(N-1)}
$$

\begin{exempel}{Exempel: Covidsjuka}
	Antag att man har testat en grupp med 20 personer för SARS-Cov-2, och att 5 av dem hade ett positivt testresultat. Antag att 6 slumpvis valda personer ur gruppen träffas på ett kafe, och låt $X$ vara antalet bland dem har viruset.\begin{enumerate}
		\item Bestäm sannolikhetsfunktionen till $X$
		\item Beräkna väntevärde och varians till $X$
		\item Beräkna sannolikheten att minst en av de som träffas har viruset
	\end{enumerate}
	\textbf{Lösning:}\\
	$$
		\text{Sannolikhetsfunktion: }P(X=k)=\frac{(^5_k)(^{15}_{6-k})}{(^{20}_{6})}
	$$$$
		\begin{cases}
			\text{Väntevärde: }\mathbb{E}[X]=\frac{5\cdot6}{20}=1.5 \\
			\text{Varians: }\text{Var}(X)=\frac{5\cdot6\cdot15\cdot14}{20^2\cdot19}=\frac{2^2\cdot7}{4\cdot19}\approx\frac{3}{4}
		\end{cases}
	$$$$
		P(X\geq1)=1-P(X-0)=\frac{(^5_0)(^{15}_6)}{(^{20}_6)}
	$$
\end{exempel}

\section{Poissonfördelning}
Poissonfördelning räknar antalet händelser $X$ som inträffar under ett tidsintervall oberoende av varandra, om det förväntade antalet händelser $\lambda>0$ är känt. Är $n$ stort och $np=\lambda$, så borde vi ha $X\approx\text{bin}(n,p)$\\
Sannolikhetsfunktionen för enn poissonfördelning ges av:$$
	f(x)=\frac{\lambda^ke^{-\lambda}}{k!}\text{ , }k\in\{1,2,\ldots\}
$$$\lambda$(intensiteten för $X$) beskriver antalet händelser per tidsenhet. Vi skriver $X\sim\text{Poisson}(\lambda)$

\section{Exponentialfördelning}
Antag att antalet händelser i intervallet $[0,1]$ är $X\sim\text{Poisson}(\lambda)$. $T$ är tiden vi väntar innan någonting händer. Låt $t>0$ och låt $X_t$ vara antalet händelser mellan $[0,t]$. Då är $X_t\sim\text{Poisson}(\lambda t)$. Vi får $$
	F_T(t)=P(T\leq t)=P(X_t>0)=1-P(X_t=0)=1-\frac{e^{-\lambda t}\lambda^0}{0!}=1-e^{-\lambda t}
$$
\subsection{Fördelningsfunktion och täthetsfunktion}
$F(t)=P(T\leq t)=-e^{-\lambda t}$ och $f(t)=F'(t)=\lambda e^{-\lambda t}$ , $t>0$
\subsection{Väntevärde, varians och momentgenererade funktioner}
$$
	\begin{cases}
		\text{Väntevärde: }\mathbb{E}[T]=\frac{1}{\lambda} \\
		\text{Varians: }\text{Var}(T)=\frac{1}{\lambda^2}  \\
		\text{Momentgenererade funktionen: } m_T(t)=\frac{1}{1-t/\lambda}
	\end{cases}
$$
\subsection{Minneslöshet}
$$
	P(T\leq s+t | X\geq t)=\frac{e^{-\lambda(s+t)}}{e^{-\lambda t}}=P(T\geq s)
$$

\section{$\chi^2$-fördelning}
$Y=X^2_1+\ldots+X^2_n\sim\chi^2_n$ har en $\chi^2$-fördelning med $n$ frihetsgrader, och vi skriver $Y\sim\chi^2_n$
\section{Gamma-fördelning}
$$\Gamma(\alpha)=\int_0^\infty z^{\alpha-1}e^{-z}dz$$
Man kan kontrollera att \begin{itemize}
	\item $\Gamma(1)=1$
	\item $\Gamma(\alpha)=(\alpha-1)\Gamma(\alpha-1)$
	\item $\Gamma(n)=(n-1)!$
\end{itemize}
