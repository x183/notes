\chapter{Models of concurrency and synchronization algorithms}
\section{Analyzing concurrency}
We can use state/transition diagrams to model elements of concurrent programs. States in a diagram capture possible program states, and transitions capture possible state changes. The following diagram models a simple program that reads a value from a shared variable, increments it, and writes it back to the variable.
\subsection{States}
A state captures the shared and local states of a concurrent program. When state is unambiguous, we simplify a state with only the value.
\subsection{Transition}
A transition is the execution leading to a state change.
\subsection{Program properties}
Mutual exclusion, Deadlock freedom, Starvation freedom, No race conditions.
\section{Mutual exclusion with only atomic reads}
Locks is a data structure used to control which thread has access to some value.\\
Can we implement lock using only atomic instructions - reading and writing shared variables?\\
\indent It's possible, but quite tricky. The easiest way is to have some kind of a lock queue of threads waiting for the lock.\\
\pagebreak
\section{Peterson's Algorithm}
A simple way of locking some memory from all but one thread without having to use a lock class.
