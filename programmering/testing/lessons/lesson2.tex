\chapter{Quality Scenarios}
\section{Scenarios}
A scenario is a description of an interaction between external stuff and the system. It defines:
\begin{itemize}
	\item An event that triggers the scenario.
	\item An interaction initiated by the external stuff.
	\item What response is required.
\end{itemize}
This is similar to use cases or user stories, but examines \textbf{both} quality and functionality.\\
\\
We use scenarios to capture a wide range of requirements:\begin{itemize}
	\item A set of interactions with users to which a system must respond
	\item Processing in response to timed events
	\item Peak load situations
	\item Regulator demands
	\item Failure response
	\item Maintainer changes stuff
	\item Whatever the design is required to handle
\end{itemize}
\subsection{Scenario usages}
Scenarios are used to; provide input to architecture definitions, by helping fleshing out your requirements and finding new ones; Evaluate system architecture, by finding missing or incompatible interfaces; communicating with stakeholders, gives easy to understand examples of what the system can do; driving the testing process by helping prioritizing your testing.

\subsection{Scenario format}
\subsubsection{Overview}
A brief description of what the scenario is.
\subsubsection{External stimulus}
What initiated the scenario? A user request, a timer, a failure, a maintainer, etc.
\subsubsection{System state}
Aspects of the system's internal state that might affect quality, such as does data exist already? is the system under high or low load, and so on.
\subsubsection{System environment}
How does the environment around the system look? How does our infrastructure look? Does the system have gigabit ethernet? is the system air gapped?
\subsubsection{System response}
How does the system respond to our stimuli? How should the system respond to meet our quality requirements?
\subsubsection{Response measure}
How does we judge the quality of the response?

\subsection{Response measure}
Most quality measurements are non-deterministic. All time-based measures should be probabilistic (95\% of the time the response should be N, 99\% of the time it's M).\\
\\
For real-time systems all time measurements should give a worst-case response time. Otherwise you can give an absolute threshold.
\subsection{What do we do with scenarios?}
We use them primarily to test and improve our design. It's also a great way of showing to stakeholders what the system can do and how it can be used.\\
\\
We also use scenarios for exploratory testing, where humans are used to test the system. We also use them for formal test cases.
\subsection{What is a good scenario?}
credible, valuable, specific, precise, comprehensive
\begin{itemize}
	\item credible
	      \subitem it should be a realistic scenario
	\item valuable
	      \subitem
	\item precise
	\item specific
	\item comprehensive
\end{itemize}

\subsection{Effective scenario use}
Identify a focused set of scenarios that are representative of the system's behavior. Too many scenarios can be a hindrance, no more than 20 scenarios.\\
\\
Use distinct scenarios, it's much better to have wildly different scenarios than to have a few scenarios that are very similar.\\
\\
Use scenarios early, they are most important in the early phase, later on it can be hard to change the system behaviour after a new set of scenarios.

\subsection{Reliability scenarios}
The ability to minimize the number of observed failures. These resolve around one function accessed through an interface.
\subsubsection{Reliability scenario format}
\begin{itemize}
	\item Overview
	      \subitem Highlight the function(s) being used and which context they are used in
	\item System state
	      \subitem Data stored or past events may impact our reliability
	\item Environment state
	      \subitem Our available resources can change how reliable our system is, an underpowered processor may cause a system to fail.
	\item External stimulus
	      \subitem A user or external system performs one or more input actions
	      \subitem State specific interactions performed
	      \subitem If relevant, the type of user and how they perceive reliability
	\item Required response
	      \subitem The functional response of the system
	\item Reliability measure
	      \subitem how should we measure the reliability of the system and how reliable should it be?
\end{itemize}
\subsection{Availability scenarios}
Ability of the system to mask or repair failures such that the outage period does not exceed a required value over a time period. In other words, we look at how the system responds to a failure, how long it takes, and what it does to return to normal operation.\\
\\
We need to distinguish between failures and the software's perception of failure, e.g. when the network goes down, the system doesn't count that as a failure until it needs to access the network, meaning we have time to recover.\\
\\
Scenarios tends to deal with; failure of internal components or external systems, reconfigure of the physical system, or maintenance/reconfigure of the software.\\
\\
\subsubsection{Availability scenario format}
\begin{itemize}
	\item Overview
	      \subitem Highlight the function(s) being used and which context they are used in.
	\item System/environment state
	\item External stimuli
	      \subitem stimuli is omissions, crashes, wrongful timings, incorrect responses and so on.
	\item Required response
	      \subitem Failure needs to be detected and isolated before we can recover.
	\item Response measure
	      \subitem Can specify an availability percentage
	      \subitem Can specify a time to detect/repair or when it needs to be available and for how long
\end{itemize}
\subsection{Availability vs Reliability}
Reliability is about how it operates normally and how often it fails. Availability is about how it operates when it fails and how we avoid hard crashes and downtime.
